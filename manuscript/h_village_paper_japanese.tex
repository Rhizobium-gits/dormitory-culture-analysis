%%%%%%%%%%%%%%%%% DO NOT CHANGE HERE %%%%%%%%%%%%%%%%%%%% 
%%%%%%%%%%%%%%%%%%%%%%%%%%%%%%%%%%%%%%%%%%%%%%%%%%%%%%%%%%{
  \documentclass[twoside,11pt]{article}
  %%%%% PACKAGES %%%%%%
  \usepackage{pgm2016}
  \usepackage{amsmath}
  \usepackage{algorithm}
  \usepackage[noend]{algpseudocode}
  \usepackage{subcaption}
  \usepackage[english]{babel}	
  \usepackage{paralist}	
  \usepackage[lowtilde]{url}
  \usepackage{fixltx2e}
  \usepackage{listings}
  \usepackage{color}
  \usepackage{hyperref}
  
  \usepackage{auto-pst-pdf}
  \usepackage{pst-all}
  \usepackage{pstricks-add}
  
  \usepackage{graphicx}
  \usepackage{float}
  
  %%%%% MACROS %%%%%%
  \algrenewcommand\Return{\State \algorithmicreturn{} }
  \algnewcommand{\LineComment}[1]{\State \(\triangleright\) #1}
    \renewcommand{\thesubfigure}{\roman{subfigure}}
    \definecolor{codegreen}{rgb}{0,0.6,0}
    \definecolor{codegray}{rgb}{0.5,0.5,0.5}
    \definecolor{codepurple}{rgb}{0.58,0,0.82}
    \definecolor{backcolour}{rgb}{0.95,0.95,0.92}
    \lstdefinestyle{mystyle}{
      backgroundcolor=\color{backcolour},  
      commentstyle=\color{codegreen},
      keywordstyle=\color{magenta},
      numberstyle=\tiny\color{codegray},
      stringstyle=\color{codepurple},
      basicstyle=\footnotesize,
      breakatwhitespace=false,        
      breaklines=true,                
      captionpos=b,                    
      keepspaces=true,                
      numbers=left,                    
      numbersep=5pt,                  
      showspaces=false,                
      showstringspaces=false,
      showtabs=false,                  
      tabsize=2
    }
    \lstset{style=mystyle}
    %%%%%%%%%%%%%%%%%%%%%%%%%%%%%%%%%%%%%%%%%%%%%%%%%%%%%%%%%% 
      %%%%%%%%%%%%%%%%%%%%%%%%%%%%%%%%%%%%%%%%%%%%%%%%%%%%%%%%%% }
  
  %%%%%%%%%%%%%%%%%%%%%%%% CHANGE HERE %%%%%%%%%%%%%%%%%%%% 
    %%%%%%%%%%%%%%%%%%%%%%%%%%%%%%%%%%%%%%%%%%%%%%%%%%%%%%%%%% {
      \newcommand\course{H-village Research}
      \newcommand\courseName{Dormitory Culture Analysis}
      \newcommand\semester{2025}
      \newcommand\assignmentNumber{}
      \newcommand\studentName{Tsubasa Sato}
      \newcommand\studentEmail{satotsubasa@keio.jp}
      \newcommand\studentNumber{Keio University}
      %%%%%%%%%%%%%%%%%%%%%%%%%%%%%%%%%%%%%%%%%%%%%%%%%%%%%%%%%% }
  %%%%%%%%%%%%%%%%%%%%%%%%%%%%%%%%%%%%%%%%%%%%%%%%%%%%%%%%%%
    
    \usepackage{CJKutf8}
    
    \newcommand{\safeincludegraphics}[2][]{%
      \IfFileExists{\detokenize{#2}}%
        {\includegraphics[#1]{\detokenize{#2}}}%
          {\fbox{Missing file: \detokenize{#2}}}%
          }
              
              %%%%%%%%%%%%%%%%% DO NOT CHANGE HERE %%%%%%%%%%%%%%%%%%%% 
              %%%%%%%%%%%%%%%%%%%%%%%%%%%%%%%%%%%%%%%%%%%%%%%%%%%%%%%%%%
                %{
                  \ShortHeadings{Keio University - H-village Dormitory Culture Study}{\studentName - \studentNumber}
                  \firstpageno{1}
                  
                  \begin{document}
                  \begin{CJK}{UTF8}{min}
                  
                  %%%%%%%%%%%%%%%%%%%%%%%%%%%%%%%%%%%%%%%%%%%%%%%%%%%%%%%%%%
                  % Title
                  %%%%%%%%%%%%%%%%%%%%%%%%%%%%%%%%%%%%%%%%%%%%%%%%%%%%%%%%%%
                  
                  \title{大学寮における居住形態と棟文化の形成:\\性別構成・空間デザインが寮生の協調性に与える影響の定量的分析}
                  
                  \author{\name \studentName \email \studentEmail \\
                    \studentNumber \\
                    Faculty of Environment and Information Studies
                  }
                  
                  \maketitle
                  
                  %%%%%%%%%%%%%%%%%%%%%%%%%%%%%%%%%%%%%%%%%%%%%%%%%%%%%%%%%%
                  % Abstract
                  %%%%%%%%%%%%%%%%%%%%%%%%%%%%%%%%%%%%%%%%%%%%%%%%%%%%%%%%%%
                    
                  \begin{abstract}
                  本研究は,慶應義塾大学SFCの学生寮「H village」を対象に,異なる性別構成(男子寮・女子寮・男女共存寮)と空間デザイン(同階共存・階分離)を持つ4棟が寮生の協調性・価値観に与える影響を検討した.2025年5月から11月にかけて計5回の縦断調査を実施し,国内学生153名のデータを分析した.棟の文化的影響を混合効果モデルによる級内相関係数(ICC)で推定した結果,全体ICCは0.075(95\% CI: [0.000, 0.258])であり,回答分散の約7.5\%が棟の違いにより説明された.「寮イベントへの参加意欲」では ICC = 0.279($p < 0.001$)であった.男女共存棟における新入寮生の文化的類似度を分析した結果,同性先輩との類似度(95.1\%)と異性先輩との類似度(95.7\%)に有意差は認められなかった($p = 0.160$).一方,男女共存棟は単一性別棟と比較してイベント参加意欲が有意に低かった($p < 0.05$).
                  \end{abstract}
                  
                  %%%%%%%%%%%%%%%%%%%%%%%%%%%%%%%%%%%%%%%%%%%%%%%%%%%%%%%%%%
                  % Keywords
                  %%%%%%%%%%%%%%%%%%%%%%%%%%%%%%%%%%%%%%%%%%%%%%%%%%%%%%%%%%
                  
                  \textbf{キーワード:} 大学寮,居住形態,性別構成,棟文化,級内相関係数(ICC),文化的類似度
                  
                  %%%%%%%%%%%%%%%%%%%%%%%%%%%%%%%%%%%%%%%%%%%%%%%%%%%%%%%%%%
                  % 1. Introduction
                  %%%%%%%%%%%%%%%%%%%%%%%%%%%%%%%%%%%%%%%%%%%%%%%%%%%%%%%%%%
                    
                  \section{はじめに}
                  
                  大学寮は学生の居住施設であると同時に,社会性発達や協調性形成の場として機能する \cite{pascarella2005, astin1993}.寮の物理的デザインや性別構成が寮生の社会性にどのような影響を与えるかについては,実証的知見の蓄積が求められている.
                  
                  日本の大学寮では,男子寮・女子寮・男女共存寮といった異なる性別構成が混在することが多い.しかし,これらの居住形態が寮生の協調性や価値観に与える影響を同一キャンパス内で比較した研究は限られている.また,男女共存寮において新入寮生が先輩寮生から文化を学習する際,同性の先輩と異性の先輩のどちらからより強い影響を受けるかという問いも検討されていない.
                  
                  本研究では,慶應義塾大学湘南藤沢キャンパス(SFC)に隣接する学生寮「H village」を対象に,居住形態と寮生の協調性の関係を検討する.
                  
                  %%%%%%%%%%%%%%%%%%%%%%%%%%%%%%%%%%%%%%%%%%%%%%%%%%%%%%%%%%
                  % 2. Theoretical Background / Literature Review
                  %%%%%%%%%%%%%%%%%%%%%%%%%%%%%%%%%%%%%%%%%%%%%%%%%%%%%%%%%%
                  
                  \section{理論的背景}
                  
                  \subsection{大学寮と学生の発達}
                  
                  Pascarella \& Terenzini (2005) は,大学寮への居住が学生の認知的・社会的発達に正の影響を与えることを報告している \cite{pascarella2005}.Astin (1993) は,学生の関与(involvement)が学習成果に影響することを示し,居住環境がその関与を促進する可能性を指摘した \cite{astin1993}.近年では,「ラーニングコミュニティ」や「リビングラーニング」の概念が提唱され,居住空間を学習・成長の場として活用する試みが報告されている \cite{inkelas2007}.
                  
                  \subsection{社会的学習理論と同性モデリング}
                  
                  Bandura (1977) の社会的学習理論によれば,人は観察学習を通じて行動や態度を獲得する \cite{bandura1977}.この理論では,観察者はモデルとの類似性が高いほどモデルの行動を模倣しやすいとされる.Bussey \& Bandura (1999) は,ジェンダー発達において同性モデルからの学習が優先される傾向を報告している \cite{bussey1999}.ただし,この同性モデリング選好が成人期の寮環境においても観察されるかは明らかではない.
                  
                  \subsection{男女共存寮に関する先行研究}
                  
                  Willoughby \& Carroll (2009) は,男女共存寮の居住者がリスク行動を取る傾向が高いことを報告している \cite{willoughby2009}.一方,男女共存寮における異性間の文化的影響のパターンについては,実証的検討が不足している.
                  
                  \subsection{空間デザインと社会的相互作用}
                  
                  Hillier \& Hanson (1984) は,建築空間の構成が人々の相互作用パターンに影響を与えることを示した \cite{hillier1984}.寮の空間デザイン(同階共存型・階分離型など)が寮生の社会的行動に与える影響については,体系的な検討が必要である.
                  
                  %%%%%%%%%%%%%%%%%%%%%%%%%%%%%%%%%%%%%%%%%%%%%%%%%%%%%%%%%%
                  % 3. Research Questions / Hypotheses
                  %%%%%%%%%%%%%%%%%%%%%%%%%%%%%%%%%%%%%%%%%%%%%%%%%%%%%%%%%%
                  
                  \section{リサーチクエスチョン}
                  
                  本研究では以下の5つのリサーチクエスチョン(RQ)を設定する.
                  
                  RQ1では,棟が寮生の協調性・価値観に関する回答パターンに有意な分散を説明するか(ICC $> 0$)を検討する.
                  
                  RQ2では,棟の影響が質問項目によって異なるかを検討する.
                  
                  RQ3では,新入寮生が時間経過とともに所属棟の文化に収束するかを検討する.
                  
                  RQ4では,男女共存棟において新入寮生が同性・異性どちらの先輩寮生からより強い影響を受けるかを検討する.社会的学習理論に基づけば,同性先輩からの影響がより強いと予測される.
                  
                  RQ5では,単一性別棟と男女共存棟で社会参加意欲や棟愛着に差があるかを検討する.
                  
                  %%%%%%%%%%%%%%%%%%%%%%%%%%%%%%%%%%%%%%%%%%%%%%%%%%%%%%%%%%
                  % 4. Data and Methods
                  %%%%%%%%%%%%%%%%%%%%%%%%%%%%%%%%%%%%%%%%%%%%%%%%%%%%%%%%%%
                    
                  \section{データと方法}
                  
                  \subsection{研究対象}
                  
                  H villageは,異なる居住形態を持つ4棟から構成される(図\ref{fig:dormitory},表\ref{tab:building_types}).Basil棟は全階が男子のみで構成される男子寮である.Turmeric棟は全階が女子のみで構成される女子寮である.Rosemary棟は同じ階に男女が居住する同階共存型の男女共存寮である.Paprika棟は2階までが男子,3--4階が女子という階分離型の男女共存寮である.
                  
                  \begin{figure}[H]
                  \centering
                  \fbox{\parbox{0.9\textwidth}{\centering\vspace{2cm}【Figure 1: H village 各棟の写真】\\(手動で挿入)\vspace{2cm}}}
                  \caption{H village の外観と各棟の配置.4棟(Basil,Turmeric,Rosemary,Paprika)の位置関係を示す.各棟は独自の居住形態を持ち,中央の共有スペースを通じて棟間の移動が可能である.}
                  \label{fig:dormitory}
                  \end{figure}
                  
                  \begin{table}[!htb]
                  \centering
                  \caption{H village 各棟の居住形態}
                  \begin{tabular}{llll}
                  \hline
                  棟名 & 性別構成 & 空間デザイン & 特徴 \\
                  \hline
                  Basil & 男子寮 & 全階男子 & 同性のみ \\
                  Turmeric & 女子寮 & 全階女子 & 同性のみ \\
                  Rosemary & 男女共存 & 同じ階に男女が居住 & 同階共存 \\
                  Paprika & 男女共存 & 2階まで男子,3-4階女子 & 階分離 \\
                  \hline
                  \end{tabular}
                  \label{tab:building_types}
                  \end{table}
                  
                  \subsection{調査対象とサンプルサイズ}
                  
                  調査対象は,H villageに居住する全学生である.2025年5月,6月,7月,10月,11月の計5回にわたり縦断調査を実施した.8月・9月は夏季休暇のため調査を実施しなかった.
                  
                  留学生は文化的差異を統制するため分析から除外し,最終的な分析対象は国内学生153名であった(表\ref{tab:sample_size}).棟別の内訳は,Basil(男子寮)が53名(34.6\%),Turmeric(女子寮)が29名(19.0\%),Rosemary(同階共存)が29名(19.0\%),Paprika(階分離)が42名(27.5\%)であった.学部1年生を「新入寮生」,学部2年生以上を「先輩寮生」として分類した.
                  
                  \begin{table}[!htb]
                  \centering
                  \caption{サンプルサイズ(棟別)}
                  \begin{tabular}{lrr}
                  \hline
                  棟 & 人数 & 割合 \\
                  \hline
                  Basil(男子寮) & 53 & 34.6\% \\
                  Turmeric(女子寮) & 29 & 19.0\% \\
                  Rosemary(同階共存) & 29 & 19.0\% \\
                  Paprika(階分離) & 42 & 27.5\% \\
                  \hline
                  計 & 153 & 100\% \\
                  \hline
                  \end{tabular}
                  \label{tab:sample_size}
                  \end{table}
                  
                  \subsection{調査項目}
                  
                  協調性・共同生活に対する考えに関する8項目について,5段階リッカート尺度(1: 全くそう思わない 〜 5: 非常にそう思う)で回答を求めた(表\ref{tab:items}).
                  
                  \begin{table}[!htb]
                  \centering
                  \caption{調査項目}
                  \begin{tabular}{lll}
                  \hline
                  項目 & 内容 & 分類 \\
                  \hline
                  Q1 & 寮内の人とコミュニケーションが取りやすい & 社会性 \\
                  Q2 & 異性との会話に障壁を感じる & 対人関係 \\
                  Q3 & ユニット内での友人関係は良好である & 親密性 \\
                  Q4 & 部屋は清潔に保っている & 個人習慣 \\
                  Q5 & 寮のイベントに積極的に参加したい & 集団参加 \\
                  Q6 & 人前での発表・プレゼンに関心がある & 自己表現 \\
                  Q7 & 自分のことを他人に打ち明けられる & 開放性 \\
                  Q8 & 自分の棟に愛着を持っている & 帰属意識 \\
                  \hline
                  \end{tabular}
                  \label{tab:items}
                  \end{table}
                  
                  \subsection{倫理的配慮}
                  
                  本調査は以下の同意文に同意した寮生のみを対象として実施した:「H-villageにおける,イベント運営や生活の改善や理解を深めるために活用するため,各棟のH生に5段階評価型のアンケートを毎月実施します.本アンケートの回答は統計的に処理され,H-villageにおけるイベント運営や生活の改善に役立てることを目的としています.集計・分析結果については今年度末に公表される予定ですが,特定の個人が識別できる情報として公表されることはありません.」
                  
                  \subsection{統計解析}
                  
                  \subsubsection{棟の文化的影響の推定(ICC)}
                  
                  棟の文化的影響は,混合効果モデル(Linear Mixed-Effects Model)により推定した \cite{bates2015}.各質問項目について,以下のNull Modelを適用した:
                  \begin{align}
                  y_{ij} = \beta_0 + u_j + \epsilon_{ij}
                  \end{align}
                  ここで,$y_{ij}$は棟$j$に所属する個人$i$のスコア,$\beta_0$は全体平均,$u_j \sim N(0, \sigma^2_b)$は棟のランダム効果,$\epsilon_{ij} \sim N(0, \sigma^2_e)$は残差である.級内相関係数(ICC)は以下により算出した:
                  \begin{align}
                  \mathrm{ICC} = \frac{\sigma^2_b}{\sigma^2_b + \sigma^2_e}
                  \end{align}
                  
                  \subsubsection{文化的類似度の測定}
                  
                  各回答者の8項目にわたる回答パターンを「文化的プロファイル」として扱い,先輩寮生の平均プロファイルとのコサイン類似度を算出した:
                  \begin{align}
                  \text{類似度} = \frac{\sum_{i=1}^{8} x_i y_i}{\sqrt{\sum_{i=1}^{8} x_i^2} \sqrt{\sum_{i=1}^{8} y_i^2}} \times 100\%
                  \end{align}
                  
                  \subsubsection{異性間文化的影響の分析}
                  
                  男女共存棟(RosemaryおよびPaprika)の新入寮生について,2種類の比較を行った.相対的比較では,各新入寮生の同性先輩との類似度と異性先輩との類似度を対応のあるt検定で比較した.絶対的比較では,新入寮生の性別にかかわらず,男性先輩プロファイルとの類似度と女性先輩プロファイルとの類似度を対応のあるt検定で比較した.
                  
                  \subsubsection{統計的検証}
                  
                  推定されたICCの妥当性を3つの手法で検証した.尤度比検定(LRT)では帰無仮説 $H_0: \sigma^2_b = 0$を検定した.ブートストラップ信頼区間では1000回のリサンプリングによる95\%パーセンタイルCIを算出した.PERMANOVAではAitchison距離に基づく並べ替え多変量分散分析を実施した \cite{anderson2001}.
                  
                  %%%%%%%%%%%%%%%%%%%%%%%%%%%%%%%%%%%%%%%%%%%%%%%%%%%%%%%%%%
                  % 5. Results
                  %%%%%%%%%%%%%%%%%%%%%%%%%%%%%%%%%%%%%%%%%%%%%%%%%%%%%%%%%%
                    
                  \section{結果}
                  
                  \subsection{RQ1:棟による回答分散}
                  
                  8項目の平均スコアに対して混合効果モデルを適用した結果,全体ICCは0.075であった(図\ref{fig:cultural_effects}a).この値は,回答分散の約7.5\%が棟の違いにより説明されることを意味する.Hedges \& Hedberg (2007) が報告した教育研究における学級効果の一般的な範囲(ICC = 0.05--0.20)\cite{hedges2007}と比較すると,本研究で観察された棟効果は学級効果と同程度の大きさであることが確認された.
                  
                  ブートストラップ法(1000回リサンプリング)による95\%信頼区間は[0.000, 0.258]であった(図\ref{fig:cultural_effects}b).信頼区間の下限が0を含むことから,棟効果が存在しない可能性を完全には棄却できない.一方,ブートストラップ分布の大部分が正の値に位置しており,点推定値0.075を中心とした分布が確認された.
                  
                  PERMANOVAでは棟による多変量的差異が検出された($R^2 = 0.078$, $F = 4.14$, $p = 0.001$)(図\ref{fig:cultural_effects}c).この結果は,8項目の回答パターン全体として見たとき,棟間に有意な差異が存在することを示している.一方,性別による差異は$R^2 = 0.010$, $p = 0.276$であり,有意ではなかった.図\ref{fig:cultural_effects}cの主座標分析(PCoA)プロットでは,各棟の95\%信頼楕円が部分的に重複しているものの,Basil(男子寮)とTurmeric(女子寮)の楕円が比較的離れた位置にあることが視覚的に確認できる.性別による差異が有意でなく棟による差異が有意であったことから,回答パターンの違いは性別そのものではなく,棟という社会的単位に関連していることが推察される.
                  
                  \begin{figure}[H]
                  \centering
                  \includegraphics[width=0.95\textwidth]{figures/fig2_cultural_effects.png}
                  \caption{棟文化効果の推定.(a) 質問項目別の級内相関係数(ICC).赤破線は8項目平均スコアに基づく全体ICC = 0.075を示す.(b) 1000回のブートストラップリサンプリングによる全体ICCの分布.黒実線は点推定値(ICC = 0.075),赤破線は95\%パーセンタイル信頼区間の境界([0.000, 0.258])を示す.(c) Aitchison距離に基づく主座標分析(PCoA).各点は個々の寮生を表し,色は所属棟,形状は居住者タイプを示す.楕円は各棟の95\%信頼楕円である.}
                  \label{fig:cultural_effects}
                  \end{figure}
                  
                  \subsection{RQ2:項目別の棟効果}
                  
                  各質問項目について個別にICCを推定した結果を表\ref{tab:icc}および図\ref{fig:cultural_effects}aに示す.Q5(イベント参加意欲)はICC = 0.279であり,尤度比検定で$p < 0.001$であった.この値は回答分散の約28\%が棟間差で説明されることを意味し,他の項目と比較して突出して高い.Q5は「寮のイベントに積極的に参加したい」という集団的活動への参加意欲を問う項目であり,棟ごとのイベント文化や参加規範が寮生の回答に反映されていることが推察される.
                  
                  Q3(ユニット内友人関係)はICC = 0.059,$p = 0.009$であり,有意な棟効果が検出された.Q3は「ユニット内での友人関係は良好である」という近接した居住者との関係性を問う項目であり,棟ごとのユニット構成や居住者間の交流パターンの違いが反映されている可能性がある.
                  
                  Q8(棟愛着)はICC = 0.077であったが,$p = 0.122$であり,有意水準0.05では有意ではなかった.棟への帰属意識に関しては,棟間差よりも個人差が大きいことが示唆される.
                  
                  一方,Q7(自己開示)はICC = 0.000,Q4(部屋の清潔さ)はICC = 0.013であり,いずれも棟の影響がほとんど検出されなかった.Q7は「自分のことを他人に打ち明けられる」,Q4は「部屋は清潔に保っている」という項目であり,これらは個人の性格特性や習慣を反映する項目である.棟という社会的単位よりも個人差が支配的であることは,これらの項目の性質と整合的である.
                  
                  以上の結果から,集団的・社会的な性質を持つ項目(Q5,Q3)では棟効果が検出されやすく,個人的な性格特性や習慣を反映する項目(Q7,Q4)では棟効果が小さいというパターンが観察された.
                  
                  \begin{table}[!htb]
                  \centering
                  \caption{質問項目別のICC・尤度比検定結果}
                  \begin{tabular}{llrrr}
                  \hline
                  項目 & 内容 & ICC & LRT $\chi^2$ & $p$値 \\
                  \hline
                  Q5 & イベント参加意欲 & 0.279 & 19.93 & $<$0.001 \\
                  Q8 & 棟への愛着 & 0.077 & 1.36 & 0.122 \\
                  Q3 & ユニット友人関係 & 0.059 & 5.55 & 0.009 \\
                  Q6 & プレゼンへの関心 & 0.039 & 2.27 & 0.066 \\
                  Q2 & 異性との障壁 & 0.028 & 0.03 & 0.433 \\
                  Q1 & コミュニケーション & 0.021 & 0.00 & 0.500 \\
                  Q4 & 部屋の清潔さ & 0.013 & 0.00 & 0.500 \\
                  Q7 & 自己開示 & 0.000 & 0.00 & 0.500 \\
                  \hline
                  \end{tabular}
                  \label{tab:icc}
                  \end{table}
                  
                  \subsection{RQ3:文化的収束}
                  
                  各寮生と所属棟の先輩寮生プロファイルとのコサイン類似度を算出した結果,全体の平均類似度は93.5\%(SD = 3.2\%)であった(図\ref{fig:convergence}a).図\ref{fig:convergence}aのバイオリンプロットに示されるように,全棟で90\%以上の高い類似度が観察され,各棟内での文化的プロファイルの一貫性が確認された.棟間で類似度の分布に大きな差は見られず,いずれの棟においても寮生は所属棟の先輩プロファイルと高い類似度を示した.
                  
                  新入寮生と先輩寮生の類似度をWilcoxon順位和検定で比較した結果,いずれの棟においても有意差は認められなかった(図\ref{fig:convergence}b).この結果は,新入寮生が入寮時点で既に先輩寮生と同程度の文化的プロファイルを持っていることを示している.2つの解釈が考えられる.第一に,新入寮生が入寮後の早い段階で棟文化に適応している可能性がある.第二に,棟の選択過程において類似した価値観を持つ学生が同じ棟に集まる自己選択効果が働いている可能性がある.本研究のデザインでは,これら2つの可能性を区別することはできない.
                  
                  時系列的な変化を図\ref{fig:convergence}cに示す.5月から11月にかけて,新入寮生・先輩寮生ともに類似度は安定しており,調査期間を通じて大きな変動は観察されなかった.新入寮生の類似度が時間とともに上昇するパターン(漸進的な文化的収束)は確認されなかった.このことは,文化的適応が入寮初期(5月以前)に既に完了しているか,あるいは自己選択効果により入寮時点で既に高い類似度が達成されていることを示唆する.
                    
                  \begin{figure}[H]
                  \centering
                  \includegraphics[width=0.95\textwidth]{figures/fig3_convergence.png}
                  \caption{文化的収束の分析.(a) 棟別の先輩寮生プロファイルとの類似度分布.バイオリンプロットは分布の形状を,ボックスプロットは中央値と四分位範囲を示す.(b) 棟内での新入寮生と先輩寮生の類似度比較.各棟についてWilcoxon順位和検定を実施した結果を示す.(c) 5ヶ月間(5月,6月,7月,10月,11月)の時系列変化.エラーバーは95\%信頼区間を表す.}
                  \label{fig:convergence}
                  \end{figure}
                    
                  \subsection{RQ4:男女共存棟における異性間文化的影響}
                    
                  RosemaryとPaprikaの新入寮生41名について分析を行った(図\ref{fig:gender_detail}).
                    
                  相対的比較において,同性先輩との類似度は平均95.1\%(SD = 3.0\%),異性先輩との類似度は平均95.7\%(SD = 2.6\%)であった.対応のあるt検定の結果,$t = -1.43$,$df = 40$,$p = 0.160$であり,有意差は認められなかった.図\ref{fig:gender_detail}aの個別対応比較では,各新入寮生の同性先輩・異性先輩との類似度を線分で結んでいる.線分の方向(同性が高いか異性が高いか)には一貫したパターンが見られず,個人によって異なる方向を示している.
                  
                  棟別に見ると,Rosemary(同階共存,$n = 18$)では同性先輩との類似度95.4\%,異性先輩との類似度95.7\%であり,$p = 0.170$であった.Paprika(階分離,$n = 23$)では同性先輩との類似度95.5\%,異性先輩との類似度95.7\%であり,$p = 0.716$であった.両棟とも有意差は認められず,空間デザイン(同階共存・階分離)にかかわらず同様のパターンが観察された.
                  
                  図\ref{fig:gender_detail}bの類似度差(同性−異性)の分布を見ると,両棟とも0付近を中心とした分布を示しており,正(同性先輩との類似度が高い)と負(異性先輩との類似度が高い)がほぼ均等に分布している.このことは,集団全体として同性先輩選好も異性先輩選好も存在しないことを示している.
                    
                  絶対的比較において,男性先輩プロファイルとの類似度は平均95.0\%(SD = 3.4\%),女性先輩プロファイルとの類似度は平均95.8\%(SD = 2.0\%)であった.対応のあるt検定の結果,$t = -1.43$,$df = 40$,$p = 0.160$であり,有意差は認められなかった.この結果は,新入寮生の性別にかかわらず,男性先輩・女性先輩の両方から同程度の文化的影響を受けていることを示している.
                  
                  図\ref{fig:gender_detail}cの影響マトリックスでは,新入寮生の性別(Male/Female)×棟(Rosemary/Paprika)の各組み合わせについて,同性先輩・異性先輩への平均類似度を示している.すべてのセルで93--97\%の高い類似度が観察され,特定の組み合わせで系統的に高いまたは低い類似度を示すパターンは見られなかった.
                  
                  図\ref{fig:gender_detail}dのサブグループ別効果量では,全体(Overall),男性新入寮生(Male New),女性新入寮生(Female New),Rosemary,Paprikaのすべてのサブグループにおいて,効果量(同性−異性の平均差)の95\%信頼区間が0を含んでいる.このことは,いずれのサブグループにおいても統計的に有意な同性先輩選好が検出されなかったことを示している.
                  
                  以上の結果から,社会的学習理論から予測される同性モデリング選好は,本研究の寮環境では支持されなかった.新入寮生は同性・異性の先輩から等しく文化的影響を受けていることが示唆される.
                    
                  \begin{figure}[H]
                  \centering
                  \includegraphics[width=0.95\textwidth]{figures/fig4_gender_influence_detail.png}
                  \caption{異性間文化的影響の分析.(a) 個別対応比較.各新入寮生について,同性先輩との類似度(左)と異性先輩との類似度(右)を線分で結ぶ.線分の色は変化の方向(青:同性が高い,赤:異性が高い)を示す.(b) 類似度差(同性−異性)の分布を棟別に示す.赤破線は差が0となる位置を示す.(c) 影響マトリックス.新入寮生の性別×棟ごとに,同性先輩・異性先輩への平均類似度をヒートマップで表示した.(d) サブグループ別の効果量(同性−異性の平均差)と95\%信頼区間を示す.}
                  \label{fig:gender_detail}
                  \end{figure}
                    
                  \subsection{RQ5:単一性別棟と男女共存棟の比較}
                    
                  建物タイプ(男子寮・女子寮・男女共存)による差を一元配置分散分析で検討した(図\ref{fig:building_type}).
                  
                  Q5(イベント参加意欲)では,建物タイプ間に有意差が認められた(図\ref{fig:building_type}a).男女共存棟(平均 = 3.85)は男子寮(4.32,$p = 0.022$)および女子寮(4.62,$p = 0.001$)の両方より低いスコアを示した.この結果は,男女共存棟の寮生が単一性別棟の寮生と比較して,寮イベントへの参加意欲が低いことを示している.男女共存環境において,何らかの要因がイベント参加への動機づけを低下させている可能性がある.
                  
                  Q8(棟愛着)でも建物タイプ間に差が観察された(図\ref{fig:building_type}b).男女共存棟(3.59)は男子寮(4.28,$p = 0.001$)より低いスコアを示した.女子寮との比較では有意差は認められなかった.男女共存棟の寮生は,少なくとも男子寮の寮生と比較して,所属棟への帰属意識が低いことが示された.
                  
                  図\ref{fig:building_type}cでは,2つの男女共存棟(RosemaryとPaprika)の比較結果を示している.Q3(ユニット友人関係)ではPaprika(階分離)がRosemary(同階共存)より0.55ポイント高く($p = 0.018$),Q5(イベント参加意欲)では1.02ポイント高く($p < 0.001$),Q6(プレゼンへの関心)では0.77ポイント高かった($p = 0.015$).
                  
                  この結果は,同じ男女共存棟であっても,空間デザインによって寮生の回答パターンに差異が生じることを示している.階分離型のPaprikaは,同階共存型のRosemaryと比較して,ユニット内の友人関係,イベント参加意欲,プレゼンへの関心のいずれも高いスコアを示した.階分離により日常生活空間での異性との接触頻度が低減されることが,これらの差異に関連している可能性がある.
                    
                  \begin{figure}[H]
                  \centering
                  \includegraphics[width=0.95\textwidth]{figures/fig5_building_type_comparison.png}
                  \caption{建物タイプ間の比較.(a) Q5(イベント参加意欲)のスコア分布を建物タイプ別に示す.*$p<0.05$,**$p<0.01$,***$p<0.001$.(b) Q8(棟愛着)のスコア分布を建物タイプ別に示す.(c) 男女共存棟間(Rosemary vs Paprika)の比較.有意差が認められた項目(Q3,Q5,Q6)について示す.}
                  \label{fig:building_type}
                  \end{figure}
                    
                  \subsection{補足分析:Q2(異性コミュニケーション障壁)}
                    
                  Q2(異性との会話に障壁を感じる)について棟×性別での分析を行った(図\ref{fig:q2_barrier}).
                  
                  図\ref{fig:q2_barrier}aに示すように,Basilの男性が最も高い平均値を示した(平均 = 2.8).Basil棟は男子寮であり,日常的に異性と接触する機会が他棟より少ない.異性との接触機会の少なさが,異性コミュニケーションへの障壁感の高さと関連している可能性がある.一方,男女共存棟(RosemaryおよびPaprika)では,男女ともに個人差が大きく,分布の幅が広いことが観察された.
                  
                  図\ref{fig:q2_barrier}bに男女共存棟における時系列変化を示す.Rosemaryの男性は5月から7月にかけて低下した後,10月に上昇し,11月に再び低下するV字型のパターンを示した.夏季休暇(8--9月)後の10月に一時的な上昇が見られたことから,休暇期間中の異性接触の減少が障壁感の上昇に関連している可能性がある.
                  
                  Paprikaの女性は5月から7月にかけて上昇した後,10月に低下し,11月に再び上昇するU字型のパターンを示した.Rosemaryの男性とは異なる変動パターンであり,異性コミュニケーション障壁感の時系列変化は,棟と性別の組み合わせによって異なることが示された.これらの非線形的な変動パターンは,異性との日常的接触に対する適応過程が単純な線形的変化ではないことを示唆している.
                    
                  \begin{figure}[H]
                  \centering
                  \includegraphics[width=0.95\textwidth]{figures/fig6_q2_barrier.png}
                  \caption{Q2(異性とのコミュニケーション障壁)の分析.(a) 棟×性別でのQ2スコア分布.ボックスプロットは中央値と四分位範囲を,個々の点は各回答者のスコアを示す.(b) 男女共存棟における時系列変化.折れ線は各棟×性別の平均スコアの推移を,エラーバーは標準誤差を,点のサイズはサンプルサイズを反映する.}
                  \label{fig:q2_barrier}
                  \end{figure}
                    
                  \subsection{補足:建物タイプ別概観}
                    
                  図\ref{fig:overview}にサンプル分布,項目別スコア,ANOVA結果の概観を示す.
                  
                  図\ref{fig:overview}aのサンプル分布では,建物タイプ×性別×居住者タイプの組み合わせごとの人数を示している.男子寮は男性のみ,女子寮は女性のみで構成されている.男女共存棟は両性別を含むが,男女比は均等ではない.各セルのサンプルサイズにばらつきがあることから,一部の比較では統計的検出力が限られている可能性がある.
                  
                  図\ref{fig:overview}bの項目別平均スコアでは,建物タイプごとの8項目の平均値と標準誤差を示している.Q5(イベント参加意欲)およびQ8(棟愛着)において,男女共存棟のスコアが単一性別棟より低いことが視覚的に確認できる.他の項目では建物タイプ間の差は比較的小さい.
                  
                  図\ref{fig:overview}cの一元配置分散分析のF統計量では,各項目に対する建物タイプ効果の強さを示している.赤線は有意水準$\alpha = 0.05$での棄却値($F_{critical}$)を示す.Q5およびQ8がこの棄却値を超えており,有意な建物タイプ効果を持つことが確認された.他の項目(Q1--Q4,Q6,Q7)はいずれも棄却値を下回っており,建物タイプ間の差は有意ではなかった.
                    
                  \begin{figure}[H]
                  \centering
                  \includegraphics[width=0.95\textwidth]{figures/fig7_building_type_overview.png}
                  \caption{建物タイプ別概観.(a) サンプル分布.建物タイプ×性別×居住者タイプの組み合わせごとの人数を示す.(b) 項目別平均スコアを建物タイプ別に示す.エラーバーは標準誤差を示す.(c) 一元配置分散分析のF統計量を各質問項目について示す.赤線は有意水準$\alpha = 0.05$での棄却値を示す.}
                  \label{fig:overview}
                  \end{figure}
                    
                  %%%%%%%%%%%%%%%%%%%%%%%%%%%%%%%%%%%%%%%%%%%%%%%%%%%%%%%%%%
                  % 6. Discussion
                  %%%%%%%%%%%%%%%%%%%%%%%%%%%%%%%%%%%%%%%%%%%%%%%%%%%%%%%%%%
                      
                  \section{考察}
                    
                  \subsection{棟文化の存在}
                  
                  全体ICCは0.075であり,回答分散の約7.5\%が棟の違いにより説明された(図\ref{fig:cultural_effects}a, b).この値はHedges \& Hedberg (2007) が報告した教育研究における学級効果の範囲(ICC = 0.05--0.20)と同程度である \cite{hedges2007}.PERMANOVAにおいて棟による差異が有意($p = 0.001$)であった一方,性別による差異は有意ではなかった($p = 0.276$)(図\ref{fig:cultural_effects}c).この結果は,回答パターンの違いが性別そのものではなく,棟という社会的単位に関連していることを示している.
                  
                  項目別では,Q5(イベント参加意欲)でICC = 0.279と最も高く(図\ref{fig:cultural_effects}a,表\ref{tab:icc}),Q7(自己開示)やQ4(部屋の清潔さ)ではICC $\approx$ 0であった.集団的活動への参加意欲を問う項目で棟効果が大きく,個人的な性格特性を反映する項目で棟効果が小さいというパターンは,棟が社会的規範の形成に関与していることと整合的である.
                    
                  \subsection{異性間文化的影響}
                  
                  男女共存棟の新入寮生において,同性先輩との類似度(95.1\%)と異性先輩との類似度(95.7\%)に有意差は認められなかった($p = 0.160$)(図\ref{fig:gender_detail}).このパターンはRosemary($p = 0.170$)とPaprika($p = 0.716$)の両棟で一貫しており,すべてのサブグループで効果量の95\%信頼区間が0を含んでいた(図\ref{fig:gender_detail}d).
                  
                  この結果は,Bandura (1977) の社会的学習理論から予測される同性モデリング選好とは異なるパターンを示している \cite{bandura1977}.Bussey \& Bandura (1999) は,ジェンダーに基づくモデル選好が文脈依存的であることを指摘しており \cite{bussey1999},本研究で測定した協調性や社会的態度がジェンダー役割と強く関連しない領域である可能性がある.また,本研究の対象者が幼少期から男女共学環境で教育を受けてきた大学生であることも,同性モデリング選好が観察されなかった要因として考えられる.
                    
                  \subsection{単一性別棟と男女共存棟の差異}
                  
                  男女共存棟は単一性別棟と比較して,Q5(イベント参加意欲)とQ8(棟愛着)で低いスコアを示した(図\ref{fig:building_type}a, b).男女共存環境において,異性の存在が社会参加を抑制する要因となっている可能性がある.評価懸念(evaluation apprehension)に関する先行研究では,潜在的な評価者の存在が個人のパフォーマンスや参加行動に影響を与えることが報告されており \cite{cottrell1972},異性の存在がこのような評価懸念を喚起している可能性がある.
                  
                  また,2つの男女共存棟の比較では,Paprika(階分離)がRosemary(同階共存)より複数の項目で高いスコアを示した(図\ref{fig:building_type}c).階分離により日常生活空間での異性との接触頻度が低減されることが,これらの差異に関連している可能性がある.この結果は,男女共存寮の空間デザインが寮生の社会参加に影響を与えることを示唆している.
                  
                  %%%%%%%%%%%%%%%%%%%%%%%%%%%%%%%%%%%%%%%%%%%%%%%%%%%%%%%%%%
                  % 7. Conclusion
                  %%%%%%%%%%%%%%%%%%%%%%%%%%%%%%%%%%%%%%%%%%%%%%%%%%%%%%%%%%
                      
                  \section{結論}
                  
                  本研究は,大学寮における棟レベルの文化的影響を混合効果モデルにより推定し,男女共存棟における異性間文化的影響のパターンを検討した.
                  
                  RQ1に関して,回答分散の約7.5\%が棟の違いで説明された(ICC = 0.075,図\ref{fig:cultural_effects}).PERMANOVAにおいて棟による差異は有意であった($p = 0.001$)一方,性別による差異は有意ではなかった($p = 0.276$).この結果から,回答パターンの違いは性別そのものではなく棟という社会的単位に関連していることが示された.
                  
                  RQ2に関して,Q5(イベント参加意欲)でICC = 0.279($p < 0.001$)であり,Q7(自己開示)やQ4(部屋の清潔さ)ではICC $\approx$ 0であった(表\ref{tab:icc},図\ref{fig:cultural_effects}a).集団的活動への参加意欲を問う項目で棟効果が大きく,個人的な性格特性を反映する項目で棟効果が小さいことが確認された.
                  
                  RQ3に関して,新入寮生と先輩寮生の類似度に有意差は認められず,新入寮生の類似度は調査期間を通じて安定していた(図\ref{fig:convergence}).この結果から,文化的適応が入寮初期に既に完了しているか,あるいは自己選択効果により入寮時点で高い類似度が達成されていることが推察された.
                  
                  RQ4に関して,男女共存棟の新入寮生において,同性先輩との類似度と異性先輩との類似度に有意差は認められなかった($p = 0.160$,図\ref{fig:gender_detail}).社会的学習理論から予測される同性モデリング選好は本研究では支持されず,新入寮生は同性・異性の先輩から等しく文化的影響を受けていることが示された.
                  
                  RQ5に関して,男女共存棟は単一性別棟よりQ5(イベント参加意欲)とQ8(棟愛着)で低いスコアを示した(図\ref{fig:building_type}a, b).また,Paprika(階分離)はRosemary(同階共存)より複数の項目で高いスコアを示した(図\ref{fig:building_type}c).この結果から,男女共存環境において何らかの要因が社会参加を抑制している可能性,および空間デザインがその抑制に影響を与える可能性が示された.
                  
                  %%%%%%%%%%%%%%%%%%%%%%%%%%%%%%%%%%%%%%%%%%%%%%%%%%%%%%%%%%
                  % 8. Limitations and Future Research
                  %%%%%%%%%%%%%%%%%%%%%%%%%%%%%%%%%%%%%%%%%%%%%%%%%%%%%%%%%%
                  
                  \section{限界と今後の課題}
                  
                  本研究の限界として以下の点が挙げられる.サンプルサイズは153名であり,特に男女共存棟の新入寮生数は41名であった.サンプルサイズの制約により,サブグループ分析における統計的検出力が限られている可能性がある(図\ref{fig:overview}a).
                  
                  本研究は単一の大学寮のみを対象としており,結果の一般化には限界がある.他大学や他国の寮環境において同様のパターンが観察されるかは不明である.
                  
                  棟の選択が無作為ではないため,選択効果を排除できない.類似した価値観を持つ学生が同じ棟を選択している可能性があり,観察された棟間差が入寮後の社会化によるものか入寮前の自己選択によるものかを区別することはできない(図\ref{fig:convergence}).
                  
                  縦断データを収集したが,本論文では主に横断的分析を行った.時系列変化の分析(図\ref{fig:convergence}c,図\ref{fig:q2_barrier}b)では5月以降のデータのみを用いており,入寮直後(4月)のデータは含まれていない.
                  
                  今後の課題として,入寮前後の変化を捉えるためのより長期的な追跡調査,他大学での比較研究,男女共存棟における社会参加意欲の差異のメカニズム解明(質的調査との組み合わせ)が挙げられる.
                    
                  %%%%%%%%%%%%%%%%%%%%%%%%%%%%%%%%%%%%%%%%%%%%%%%%%%%%%%%%%%
                  % データ出典
                  %%%%%%%%%%%%%%%%%%%%%%%%%%%%%%%%%%%%%%%%%%%%%%%%%%%%%%%%%%
                      
                  \section*{データ出典}
                  
                  H-village「各棟における性格の傾向調査」アンケート回答データ(2025年5月・6月・7月・10月・11月)を用いた.
                    
                  %%%%%%%%%%%%%%%%%%%%%%%%%%%%%%%%%%%%%%%%%%%%%%%%%%%%%%%%%%
                  % References
                  %%%%%%%%%%%%%%%%%%%%%%%%%%%%%%%%%%%%%%%%%%%%%%%%%%%%%%%%%%
                      
                  \begin{thebibliography}{13}
                    
                  \bibitem{pascarella2005}
                  Pascarella, E. T. \& Terenzini, P. T. (2005).
                  \textit{How college affects students: A third decade of research} (Vol. 2).
                  Jossey-Bass.
                    
                  \bibitem{astin1993}
                  Astin, A. W. (1993).
                  \textit{What matters in college? Four critical years revisited}.
                  Jossey-Bass.
                    
                  \bibitem{inkelas2007}
                  Inkelas, K. K., et al. (2007).
                  Living--learning programs and first-generation college students' academic and social transition to college.
                  \textit{Research in Higher Education}, 48(4), 403--434.
                    
                  \bibitem{bandura1977}
                  Bandura, A. (1977).
                  \textit{Social learning theory}.
                  Prentice-Hall.
                    
                  \bibitem{bussey1999}
                  Bussey, K. \& Bandura, A. (1999).
                  Social cognitive theory of gender development and differentiation.
                  \textit{Psychological Review}, 106(4), 676--713.
                    
                  \bibitem{bates2015}
                  Bates, D., et al. (2015).
                  Fitting linear mixed-effects models using lme4.
                  \textit{Journal of Statistical Software}, 67(1), 1--48.
                    
                  \bibitem{anderson2001}
                  Anderson, M. J. (2001).
                  A new method for non-parametric multivariate analysis of variance.
                  \textit{Austral Ecology}, 26(1), 32--46.
                    
                  \bibitem{hedges2007}
                  Hedges, L. V. \& Hedberg, E. C. (2007).
                  Intraclass correlation values for planning group-randomized trials in education.
                  \textit{Educational Evaluation and Policy Analysis}, 29(1), 60--87.
                    
                  \bibitem{willoughby2009}
                  Willoughby, B. J. \& Carroll, J. S. (2009).
                  The impact of living in co-ed resident halls on risk-taking among college students.
                  \textit{Journal of American College Health}, 58(3), 241--246.
                    
                  \bibitem{hillier1984}
                  Hillier, B. \& Hanson, J. (1984).
                  \textit{The social logic of space}.
                  Cambridge University Press.
                  
                  \bibitem{cottrell1972}
                  Cottrell, N. B. (1972).
                  Social facilitation.
                  In C. G. McClintock (Ed.), \textit{Experimental social psychology} (pp. 185--236).
                  Holt, Rinehart \& Winston.
                    
                  \bibitem{rcore2024}
                  R Core Team (2024).
                  \textit{R: A language and environment for statistical computing}.
                  R Foundation for Statistical Computing.
                    
                  \end{thebibliography}
                    
                  \end{CJK}
                  \end{document}
